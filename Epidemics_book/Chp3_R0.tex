\PassOptionsToPackage{unicode=true}{hyperref} % options for packages loaded elsewhere
\PassOptionsToPackage{hyphens}{url}
%
\documentclass[]{article}
\usepackage{lmodern}
\usepackage{amssymb,amsmath}
\usepackage{ifxetex,ifluatex}
\usepackage{fixltx2e} % provides \textsubscript
\ifnum 0\ifxetex 1\fi\ifluatex 1\fi=0 % if pdftex
  \usepackage[T1]{fontenc}
  \usepackage[utf8]{inputenc}
  \usepackage{textcomp} % provides euro and other symbols
\else % if luatex or xelatex
  \usepackage{unicode-math}
  \defaultfontfeatures{Ligatures=TeX,Scale=MatchLowercase}
\fi
% use upquote if available, for straight quotes in verbatim environments
\IfFileExists{upquote.sty}{\usepackage{upquote}}{}
% use microtype if available
\IfFileExists{microtype.sty}{%
\usepackage[]{microtype}
\UseMicrotypeSet[protrusion]{basicmath} % disable protrusion for tt fonts
}{}
\IfFileExists{parskip.sty}{%
\usepackage{parskip}
}{% else
\setlength{\parindent}{0pt}
\setlength{\parskip}{6pt plus 2pt minus 1pt}
}
\usepackage{hyperref}
\hypersetup{
            pdftitle={Chapter 3 - R0},
            pdfauthor={Pablo Perez},
            pdfborder={0 0 0},
            breaklinks=true}
\urlstyle{same}  % don't use monospace font for urls
\usepackage[margin=1in]{geometry}
\usepackage{color}
\usepackage{fancyvrb}
\newcommand{\VerbBar}{|}
\newcommand{\VERB}{\Verb[commandchars=\\\{\}]}
\DefineVerbatimEnvironment{Highlighting}{Verbatim}{commandchars=\\\{\}}
% Add ',fontsize=\small' for more characters per line
\usepackage{framed}
\definecolor{shadecolor}{RGB}{248,248,248}
\newenvironment{Shaded}{\begin{snugshade}}{\end{snugshade}}
\newcommand{\AlertTok}[1]{\textcolor[rgb]{0.94,0.16,0.16}{#1}}
\newcommand{\AnnotationTok}[1]{\textcolor[rgb]{0.56,0.35,0.01}{\textbf{\textit{#1}}}}
\newcommand{\AttributeTok}[1]{\textcolor[rgb]{0.77,0.63,0.00}{#1}}
\newcommand{\BaseNTok}[1]{\textcolor[rgb]{0.00,0.00,0.81}{#1}}
\newcommand{\BuiltInTok}[1]{#1}
\newcommand{\CharTok}[1]{\textcolor[rgb]{0.31,0.60,0.02}{#1}}
\newcommand{\CommentTok}[1]{\textcolor[rgb]{0.56,0.35,0.01}{\textit{#1}}}
\newcommand{\CommentVarTok}[1]{\textcolor[rgb]{0.56,0.35,0.01}{\textbf{\textit{#1}}}}
\newcommand{\ConstantTok}[1]{\textcolor[rgb]{0.00,0.00,0.00}{#1}}
\newcommand{\ControlFlowTok}[1]{\textcolor[rgb]{0.13,0.29,0.53}{\textbf{#1}}}
\newcommand{\DataTypeTok}[1]{\textcolor[rgb]{0.13,0.29,0.53}{#1}}
\newcommand{\DecValTok}[1]{\textcolor[rgb]{0.00,0.00,0.81}{#1}}
\newcommand{\DocumentationTok}[1]{\textcolor[rgb]{0.56,0.35,0.01}{\textbf{\textit{#1}}}}
\newcommand{\ErrorTok}[1]{\textcolor[rgb]{0.64,0.00,0.00}{\textbf{#1}}}
\newcommand{\ExtensionTok}[1]{#1}
\newcommand{\FloatTok}[1]{\textcolor[rgb]{0.00,0.00,0.81}{#1}}
\newcommand{\FunctionTok}[1]{\textcolor[rgb]{0.00,0.00,0.00}{#1}}
\newcommand{\ImportTok}[1]{#1}
\newcommand{\InformationTok}[1]{\textcolor[rgb]{0.56,0.35,0.01}{\textbf{\textit{#1}}}}
\newcommand{\KeywordTok}[1]{\textcolor[rgb]{0.13,0.29,0.53}{\textbf{#1}}}
\newcommand{\NormalTok}[1]{#1}
\newcommand{\OperatorTok}[1]{\textcolor[rgb]{0.81,0.36,0.00}{\textbf{#1}}}
\newcommand{\OtherTok}[1]{\textcolor[rgb]{0.56,0.35,0.01}{#1}}
\newcommand{\PreprocessorTok}[1]{\textcolor[rgb]{0.56,0.35,0.01}{\textit{#1}}}
\newcommand{\RegionMarkerTok}[1]{#1}
\newcommand{\SpecialCharTok}[1]{\textcolor[rgb]{0.00,0.00,0.00}{#1}}
\newcommand{\SpecialStringTok}[1]{\textcolor[rgb]{0.31,0.60,0.02}{#1}}
\newcommand{\StringTok}[1]{\textcolor[rgb]{0.31,0.60,0.02}{#1}}
\newcommand{\VariableTok}[1]{\textcolor[rgb]{0.00,0.00,0.00}{#1}}
\newcommand{\VerbatimStringTok}[1]{\textcolor[rgb]{0.31,0.60,0.02}{#1}}
\newcommand{\WarningTok}[1]{\textcolor[rgb]{0.56,0.35,0.01}{\textbf{\textit{#1}}}}
\usepackage{graphicx,grffile}
\makeatletter
\def\maxwidth{\ifdim\Gin@nat@width>\linewidth\linewidth\else\Gin@nat@width\fi}
\def\maxheight{\ifdim\Gin@nat@height>\textheight\textheight\else\Gin@nat@height\fi}
\makeatother
% Scale images if necessary, so that they will not overflow the page
% margins by default, and it is still possible to overwrite the defaults
% using explicit options in \includegraphics[width, height, ...]{}
\setkeys{Gin}{width=\maxwidth,height=\maxheight,keepaspectratio}
\setlength{\emergencystretch}{3em}  % prevent overfull lines
\providecommand{\tightlist}{%
  \setlength{\itemsep}{0pt}\setlength{\parskip}{0pt}}
\setcounter{secnumdepth}{0}
% Redefines (sub)paragraphs to behave more like sections
\ifx\paragraph\undefined\else
\let\oldparagraph\paragraph
\renewcommand{\paragraph}[1]{\oldparagraph{#1}\mbox{}}
\fi
\ifx\subparagraph\undefined\else
\let\oldsubparagraph\subparagraph
\renewcommand{\subparagraph}[1]{\oldsubparagraph{#1}\mbox{}}
\fi

% set default figure placement to htbp
\makeatletter
\def\fps@figure{htbp}
\makeatother


\title{Chapter 3 - R0}
\author{Pablo Perez}
\date{30/10/2020}

\begin{document}
\maketitle

\hypertarget{estimating-r0-from-a-simple-epidemic}{%
\subsection{Estimating R0 from a simple
epidemic}\label{estimating-r0-from-a-simple-epidemic}}

For free-living organisms, the rate of exponential growth \texttt{r}
given generation time \texttt{G} can be calculated as follows
\[r = log(R_0) / G\] \[R_0 = exp(rG)\]

Since an exponentially growing population grows as a rate
\[N(t) = N(0)exp(rt)\] the time for it to double will be \[log(2)/r\]

For pathogens, \texttt{N} would represent disease prevalence and
\texttt{G} the serial interval (i.e.~time between successive cases in a
chain of transmission). Since infectious disease data often is in the
form of \emph{incidence},
\[r = log(cumulative_incidence)\]\[R_0 = Vr + 1\]

\hypertarget{example-2003-niamey-measles-outbreak}{%
\subsubsection{Example: 2003 Niamey measles
outbreak}\label{example-2003-niamey-measles-outbreak}}

\includegraphics{Chp3_R0_files/figure-latex/unnamed-chunk-1-1.pdf}

Data in this example is weekly and for measles the serial interval is 10
to 12 days. \texttt{R\_0} can thus be calculated assuming a \texttt{V}
of 1.5 or 1.8.

\begin{Shaded}
\begin{Highlighting}[]
\NormalTok{fit=}\KeywordTok{lm}\NormalTok{(}\KeywordTok{log}\NormalTok{(cum_cases)}\OperatorTok{~}\NormalTok{absweek, }\DataTypeTok{subset=}\NormalTok{absweek}\OperatorTok{<}\DecValTok{7}\NormalTok{,}
       \DataTypeTok{data=}\NormalTok{niamey)}
\NormalTok{r=fit}\OperatorTok{$}\NormalTok{coef[}\StringTok{"absweek"}\NormalTok{]}
\NormalTok{V=}\KeywordTok{c}\NormalTok{(}\FloatTok{1.5}\NormalTok{, }\FloatTok{1.8}\NormalTok{)}
\NormalTok{V}\OperatorTok{*}\NormalTok{r}\OperatorTok{+}\DecValTok{1}
\end{Highlighting}
\end{Shaded}

\begin{verbatim}
## [1] 1.694233 1.833080
\end{verbatim}

The above estimate of \texttt{R\_0} between 1.5 and 2.0 really
represents \texttt{R\_E}, the \emph{effective} reproductive number at
the start of a measles outbreak in a population with a background level
of vaccination.

This estimate can be refined by futher acounting for the ration of
infectious period to serial interval \texttt{f}
\[R = Vr + 1 + f(1-f)(Vr)^2\]

\begin{Shaded}
\begin{Highlighting}[]
\NormalTok{V =}\StringTok{ }\KeywordTok{c}\NormalTok{(}\FloatTok{1.5}\NormalTok{, }\FloatTok{1.8}\NormalTok{)}
\NormalTok{f =}\StringTok{ }\NormalTok{(}\DecValTok{5}\OperatorTok{/}\DecValTok{7}\NormalTok{)}\OperatorTok{/}\NormalTok{V}
\NormalTok{V }\OperatorTok{*}\StringTok{ }\NormalTok{r }\OperatorTok{+}\StringTok{ }\DecValTok{1} \OperatorTok{+}\StringTok{ }\NormalTok{f }\OperatorTok{*}\StringTok{ }\NormalTok{(}\DecValTok{1} \OperatorTok{-}\StringTok{ }\NormalTok{f) }\OperatorTok{*}\StringTok{ }\NormalTok{(V }\OperatorTok{*}\StringTok{ }\NormalTok{r)}\OperatorTok{^}\DecValTok{2}
\end{Highlighting}
\end{Shaded}

\begin{verbatim}
## [1] 1.814450 1.999198
\end{verbatim}

Whilst these simple calculations of \texttt{R\_0} based on initial
growth are handy, they're not very rigorous as they rely on only a
fraction of the data.

\hypertarget{maximum-likelihood-the-chain-binomial-model}{%
\subsection{Maximum likelihood: the chain-binomial
model}\label{maximum-likelihood-the-chain-binomial-model}}

In contrast to the continuous-time, deterministic, basic-chain SIR model
from chapter 2, the \textbf{chain-binomial model} assumes that an
epidemic arises from a succession of discrete generations of infectious
individuals in a \emph{coin-flip} fashion given a force of infection
\[\beta\ I/N\]

The negative for the above expression will thus be the probability that
a \emph{susceptible} individual escapes infection at time \texttt{t}.
The number of events in \texttt{∆t} will follow a Poisson distribution
(\texttt{x∆t}). Assuming that a) contacts happen at random and b) the
serial interval is the basic unit of time \texttt{t}
\[I_{t+1} ~ Binomial(S_t, 1 - exp(-\beta\ I_t / N)\]
\[S_{t+1} = S_t - I_{t+1} = S_0 - \sum_{i=1}^t I_i\] where
\texttt{\textbackslash{}beta} approximates to the reproductive ration at
the start of the epidemic. Assuming \emph{conditional independence}
(i.e.~each epidemic generation depends only on the state of the system
in the previous time step), the removal method can estimate
\texttt{\textbackslash{}beta} and \texttt{S\_0} from a sequence of
binomial likelihoods.

\end{document}
